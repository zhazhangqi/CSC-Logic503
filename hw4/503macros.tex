% Package and macro definitions for CSC 503
% Originally prepared August 23, 2012 by Jon Doyle

%%%%%%%%%%%%%%%%%%%%%%%%%%%%%%%%%%%%%%%%%%%%%%%%%%%%%%%%%%%%%%%%%%%%%%
%%% Font and symbol definition packages
%%%%%%%%%%%%%%%%%%%%%%%%%%%%%%%%%%%%%%%%%%%%%%%%%%%%%%%%%%%%%%%%%%%%%%
\usepackage{times} 
\usepackage{helvet} 
%%% AMS MATH FONTS
\usepackage{amsmath}
\usepackage{amssymb}
%%% eucal changes \mathcal to use Euler script font
\usepackage{eucal}
%%% ST.MARY ROAD FONT
%%% stmaryrd provides program semantics symbols
%%% \llparenthesis, \rrparenthesis
\usepackage{stmaryrd}

%%% MULTIPLE COLUMNS (for typesetting answer sheets)
\usepackage{multicol}

%%%%%%%%%%%%%%%%%%%%%%%%%%%%%%%%%%%%%%%%%%%%%%%%%%%%%%%%%%%%%%%%%%%%%%
%%% Homework page dimensions
%%%%%%%%%%%%%%%%%%%%%%%%%%%%%%%%%%%%%%%%%%%%%%%%%%%%%%%%%%%%%%%%%%%%%%
\setlength{\oddsidemargin}{0in}
\setlength{\evensidemargin}{0in}
\setlength{\topmargin}{0in}
\setlength{\textheight}{9in}
\setlength{\textwidth}{6.5in}
\setlength{\headheight}{0in}
\setlength{\headsep}{0in}
\setlength{\footskip}{0.5in}


%%%%%%%%%%%%%%%%%%%%%%%%%%%%%%%%%%%%%%%%%%%%%%%%%%%%%%%%%%%%%%%%%%%%%% 
%%% Homework text style
%%%%%%%%%%%%%%%%%%%%%%%%%%%%%%%%%%%%%%%%%%%%%%%%%%%%%%%%%%%%%%%%%%%%%%
\sloppy
\raggedbottom


%%%%%%%%%%%%%%%%%%%%%%%%%%%%%%%%%%%%%%%%%%%%%%%%%%%%%%%%%%%%%%%%%%%%%%
%%% INDICATING ANSWERS TO HOMEWORK PROBLEMS
%%%%%%%%%%%%%%%%%%%%%%%%%%%%%%%%%%%%%%%%%%%%%%%%%%%%%%%%%%%%%%%%%%%%%%
\newenvironment{answer}%
{\par\noindent\textbf{Answer}\par\noindent}{}


%%%%%%%%%%%%%%%%%%%%%%%%%%%%%%%%%%%%%%%%%%%%%%%%%%%%%%%%%%%%%%%%%%%%%%
%%% PACKAGES FOR CREATING AND INCLUDING GRAPHICS
%%%%%%%%%%%%%%%%%%%%%%%%%%%%%%%%%%%%%%%%%%%%%%%%%%%%%%%%%%%%%%%%%%%%%%

%%% TIKZ DIAGRAMMING PACKAGE
\usepackage{tikz}
\usetikzlibrary{arrows,automata}

%%% GRAPHICX: IMAGE FILE INCLUSION
\usepackage{graphicx}
\DeclareGraphicsExtensions{.jpg,.jpeg,.pdf,.eps,.epsi,.wmf}


%%%%%%%%%%%%%%%%%%%%%%%%%%%%%%%%%%%%%%%%%%%%%%%%%%%%%%%%%%%%%%%%%%%%%%
%%% NUMBER SYSTEMS
%%%%%%%%%%%%%%%%%%%%%%%%%%%%%%%%%%%%%%%%%%%%%%%%%%%%%%%%%%%%%%%%%%%%%%
%%% NATURAL NUMBERS
\newcommand{\NN}{\mathord{\mathbb{N}}}
%%% INTEGERS
\newcommand{\ZN}{\mathord{\mathbb{Z}}}
%%% RATIONAL NUMBERS
\newcommand{\QN}{\mathord{\mathbb{Q}}}
%%% REAL NUMBERS
\newcommand{\RN}{\mathord{\mathbb{R}}}


%%%%%%%%%%%%%%%%%%%%%%%%%%%%%%%%%%%%%%%%%%%%%%%%%%%%%%%%%%%%%%%%%%%%%%
%%% LOGICAL LANGUAGE SYMBOLS
%%%%%%%%%%%%%%%%%%%%%%%%%%%%%%%%%%%%%%%%%%%%%%%%%%%%%%%%%%%%%%%%%%%%%%
%%% DEFINITIONS ONLY WHERE NEEDED

%%% CONNECTIVES

%%% TRUTH
%%% \top

%%% FALSITY
%%% \bot

%%% NEGATION
%%% \neg
\newcommand{\lneg}{\neg}

%%% CONJUNCTION
%%% \land
\newcommand{\bigland}{\bigwedge}

%%% DISJUNCTION
%%% \lor
\newcommand{\biglor}{\bigvee}

%%% IMPLICATION
\renewcommand{\implies}{\rightarrow}
%%% BIIMPLICATION (IF AND ONLY IF)
\renewcommand{\iff}{\leftrightarrow}

%%% QUANTIFIERS
%%% \forall
%%% \exists

%%%%%%%%%%%%%%%%%%%%%%%%%%%%%%%%%%%%%%%%%%%%%%%%%%%%%%%%%%%%%%%%%%%%%%
%%% LOGICAL METALANGUAGE SYMBOLS
%%%%%%%%%%%%%%%%%%%%%%%%%%%%%%%%%%%%%%%%%%%%%%%%%%%%%%%%%%%%%%%%%%%%%%
%%% ENTAILMENT
%%% \models
%%% EQUIVALENCE
%%% \equiv
%%% LOGICAL CONSEQUENCES
\newcommand{\Cn}{\text{Cn}}               
%%% TURNSTILE
\newcommand{\turn}{\vdash}
%%% NEGATED TURNSTILE
\newcommand{\notturn}{\nvdash}	       
%%% BIPROVABLE
\newcommand{\lrturn}{\dashv\vdash}
%%% DEDUCTIVE CLOSURE
\newcommand{\Th}{\mathord{\text{Th}}}


%%%%%%%%%%%%%%%%%%%%%%%%%%%%%%%%%%%%%%%%%%%%%%%%%%%%%%%%%%%%%%%%%%%%%%
%%% MODAL LOGIC LANGUAGE
%%%%%%%%%%%%%%%%%%%%%%%%%%%%%%%%%%%%%%%%%%%%%%%%%%%%%%%%%%%%%%%%%%%%%%

%%% MODAL LOGIC SATISFACTION/FORCING
\newcommand{\fmodels}{\Vdash}

%%% NECESSITY AND POSSIBILITY
%%% \Box
%%% \Diamond

%%% LTL TEMPORAL MODALITIES
\newcommand{\Sometime}{\mathord{\mathsf{F}}}
\newcommand{\Forever}{\mathord{\mathsf{G}}}
\newcommand{\NextO}{\mathord{\mathsf{O}}}
\newcommand{\NextX}{\mathord{\mathsf{X}}}
\newcommand{\Next}{\NextX}
\newcommand{\Until}{\mathrel{\mathsf{U}}}
\newcommand{\Release}{\mathrel{\mathsf{R}}}
\newcommand{\WeakUntil}{\mathrel{\mathsf{W}}}
\newcommand{\Before}{\mathrel{\mathsf{B}}}

%%% CTL PATH QUANTIFIERS
\newcommand{\All}{\mathord{\mathsf{A}}}
\newcommand{\Exists}{\mathord{\mathsf{E}}}

%%% EPISTEMIC LOGIC MODALITIES
%%% K is basic knowledge modality
\newcommand{\Everyone}{\mathord{\mathsf{E}}}
\newcommand{\Common}{\mathord{\mathsf{C}}}
\newcommand{\Distributed}{\mathord{\mathsf{D}}}

%%%%%%%%%%%%%%%%%%%%%%%%%%%%%%%%%%%%%%%%%%%%%%%%%%%%%%%%%%%%%%%%%%%%%%
%%% HOARE LOGIC LANGUAGE
%%%%%%%%%%%%%%%%%%%%%%%%%%%%%%%%%%%%%%%%%%%%%%%%%%%%%%%%%%%%%%%%%%%%%%
%%% HOARE PARENTHESES
\newcommand{\lHcond}{\mathopen\llparenthesis}
\newcommand{\rHcond}{\mathclose\rrparenthesis}
%%% HOARE TRIPLES
\newcommand{\Hcond}[1]{\lHcond #1 \rHcond}
\newcommand{\Hoare}[3]{\Hcond{#1} \mathrel{#2} \Hcond{#3}}

%%%%%%%%%%%%%%%%%%%%%%%%%%%%%%%%%%%%%%%%%%%%%%%%%%%%%%%%%%%%%%%%%%%%%%
%%% BASIC PROGRAMS (HUTH AND RYAN)
%%%%%%%%%%%%%%%%%%%%%%%%%%%%%%%%%%%%%%%%%%%%%%%%%%%%%%%%%%%%%%%%%%%%%%
\newcommand{\hrprogfont}{\mathtt}
\newcommand{\If}{\mathrel{\hrprogfont{if}}}
\newcommand{\Else}{\mathrel{\hrprogfont{else}}}
\newcommand{\While}{\mathrel{\hrprogfont{while}}}
\newcommand{\IfElse}[3]{\If #1 \ \{ #2 \} \Else \{ #3 \}}
\newcommand{\Whiledo}[2]{\While #1\ \{ #2 \}}
\newcommand{\True}{\mathord{\hrprogfont{true}}}
\newcommand{\False}{\mathord{\hrprogfont{false}}}

\newcommand{\pgets}{\mathrel{\mathop:}\mathord{-}}

%%%%%%%%%%%%%%%%%%%%%%%%%%%%%%%%%%%%%%%%%%%%%%%%%%%%%%%%%%%%%%%%%%%%%%
%%% PARTIAL AND TOTAL CORRECTNESS
%%%%%%%%%%%%%%%%%%%%%%%%%%%%%%%%%%%%%%%%%%%%%%%%%%%%%%%%%%%%%%%%%%%%%%
%%% PARTIAL AND TOTAL CORRECTNESS ENTAILMENT
\newcommand{\parmodels}{\mathrel{\models_{\textup{par}}}}
\newcommand{\totmodels}{\mathrel{\models_{\textup{tot}}}}
%%% PARTIAL AND TOTAL CORRECTNESS PROVABILITY
\newcommand{\parturn}{\mathrel{\turn_{\textup{par}}}}
\newcommand{\totturn}{\mathrel{\turn_{\textup{tot}}}}
%%% CONSEQUENCE OF ARITHMETIC
\newcommand{\arturn}{\mathrel{\turn_{\textup{AR}}}}

%%%%%%%%%%%%%%%%%%%%%%%%%%%%%%%%%%%%%%%%%%%%%%%%%%%%%%%%%%%%%%%%%%%%%%
%%% OTHER MACROS
%%%%%%%%%%%%%%%%%%%%%%%%%%%%%%%%%%%%%%%%%%%%%%%%%%%%%%%%%%%%%%%%%%%%%%
%%% DEFINITIONAL EQUALITY
\newcommand{\defeq}{\stackrel{\text{\tiny def}}{=}}   % Defining equality
%%% Font for code
\newcommand{\code}[1]{{\texttt{\textbf{#1}}}}
\newcommand{\Code}[1]{{\mathtt{\mathbf{#1}}}}

%%%%%%%%%%%%%%%%%%%%%%%%%%%%%%%%%%%%%%%%%%%%%%%%%%%%%%%%%%%%%%%%%%%%%%
%%% \UNITYID COMMAND FOR INDICATING AUTHORSHIP
%%% Example usage:  \unityid{jdoyle2}
%%%%%%%%%%%%%%%%%%%%%%%%%%%%%%%%%%%%%%%%%%%%%%%%%%%%%%%%%%%%%%%%%%%%%%
\newcommand{\unityid}[1]{%
  \def\testidarg{#1}%
  \def\missingidarg{MISSING}
  \def\emptyidarg{}%
  Unity ID:\ %
  \ifx\testidarg\missingidarg%
  \errmessage{unityid: Replace MISSING with your Unity ID in your
    answer file.  Hit Enter to continue.}%
  #1%
  \else%
  \ifx\testidarg\emptyidarg%
  \errmessage{unityid: Fill in your Unity ID in your answer file.  Hit
    Enter to continue.}%
  \missingidarg%
  \else%
  #1%
  \fi%
  \fi%
}

%%%%%%%%%%%%%%%%%%%%%%%%%%%%%%%%%%%%%%%%%%%%%%%%%%%%%%%%%%%%%%%%%%%%%%
%%% SELLINGER'S FITCH.STY PACKAGE FOR PROOFS
%%%%%%%%%%%%%%%%%%%%%%%%%%%%%%%%%%%%%%%%%%%%%%%%%%%%%%%%%%%%%%%%%%%%%%
\makeatletter
% Macros for Fitch-style natural deduction. 
% Author: Peter Selinger, University of Ottawa
% Created: Jan 14, 2002
% Modified: Feb 8, 2005
% Version: 0.5
% Copyright: (C) 2002-2005 Peter Selinger
% Filename: fitch.sty
% Documentation: fitchdoc.tex
% URL: http://quasar.mathstat.uottawa.ca/~selinger/fitch/

% License:
%
% This program is free software; you can redistribute it and/or modify
% it under the terms of the GNU General Public License as published by
% the Free Software Foundation; either version 2, or (at your option)
% any later version.
%
% This program is distributed in the hope that it will be useful, but
% WITHOUT ANY WARRANTY; without even the implied warranty of
% MERCHANTABILITY or FITNESS FOR A PARTICULAR PURPOSE. See the GNU
% General Public License for more details.
%
% You should have received a copy of the GNU General Public License
% along with this program; if not, write to the Free Software Foundation, 
% Inc., 59 Temple Place, Suite 330, Boston, MA 02111-1307, USA.

% USAGE EXAMPLE:
% 
% The following is a simple example illustrating the usage of this
% package.  For detailed instructions and additional functionality, see
% the user guide, which can be found in the file fitchdoc.tex.
% 
% \[
% \begin{nd}
%   \hypo{1}  {P\vee Q}   
%   \hypo{2}  {\neg Q}                         
%   \open                              
%   \hypo{3a} {P}
%   \have{3b} {P}        \r{3a}
%   \close                   
%   \open
%   \hypo{4a} {Q}
%   \have{4b} {\neg Q}   \r{2}
%   \have{4c} {\bot}     \ne{4a,4b}
%   \have{4d} {P}        \be{4c}
%   \close                             
%   \have{5}  {P}        \oe{1,3a-3b,4a-4d}                 
% \end{nd}
% \]

{\chardef\x=\catcode`\*
\catcode`\*=11
\global\let\nd*astcode\x}
\catcode`\*=11

% References

\newcount\nd*ctr
\def\nd*render{\expandafter\ifx\expandafter\nd*x\nd*base\nd*x\the\nd*ctr\else\nd*base\ifnum\nd*ctr<0\the\nd*ctr\else\ifnum\nd*ctr>0+\the\nd*ctr\fi\fi\fi}
\expandafter\def\csname nd*-\endcsname{}

\def\nd*num#1{\nd*numo{\nd*render}{#1}\global\advance\nd*ctr1}
\def\nd*numopt#1#2{\nd*numo{$#1$}{#2}}
\def\nd*numo#1#2{\edef\x{#1}\mbox{$\x$}\expandafter\global\expandafter\let\csname nd*-#2\endcsname\x}
\def\nd*ref#1{\expandafter\let\expandafter\x\csname nd*-#1\endcsname\ifx\x\relax%
  \errmessage{Undefined natdeduction reference: #1}\else\mbox{$\x$}\fi}
\def\nd*noop{}
\def\nd*set#1#2{\ifx\relax#1\nd*noop\else\global\def\nd*base{#1}\fi\ifx\relax#2\relax\else\global\nd*ctr=#2\fi}
\def\nd*reset{\nd*set{}{1}}
\def\nd*refa#1{\nd*ref{#1}}
\def\nd*aux#1#2{\ifx#2-\nd*refa{#1}--\def\nd*c{\nd*aux{}}%
  \else\ifx#2,\nd*refa{#1}, \def\nd*c{\nd*aux{}}%
  \else\ifx#2;\nd*refa{#1}; \def\nd*c{\nd*aux{}}%
  \else\ifx#2.\nd*refa{#1}. \def\nd*c{\nd*aux{}}%
  \else\ifx#2)\nd*refa{#1})\def\nd*c{\nd*aux{}}%
  \else\ifx#2(\nd*refa{#1}(\def\nd*c{\nd*aux{}}%
  \else\ifx#2\nd*end\nd*refa{#1}\def\nd*c{}%
  \else\def\nd*c{\nd*aux{#1#2}}%
  \fi\fi\fi\fi\fi\fi\fi\nd*c}
\def\ndref#1{\nd*aux{}#1\nd*end}

% Layer A

% define various dimensions (explained in fitchdoc.tex):
\newlength{\nd*dim} 
\newdimen\nd*depthdim
\newdimen\nd*hsep
\newdimen\ndindent
\ndindent=1em
% user command to redefine dimensions
\def\nddim#1#2#3#4#5#6#7#8{\nd*depthdim=#3\relax\nd*hsep=#6\relax%
\def\nd*height{#1}\def\nd*thickness{#8}\def\nd*initheight{#2}%
\def\nd*indent{#5}\def\nd*labelsep{#4}\def\nd*justsep{#7}}
% set initial dimensions
\nddim{4.5ex}{3.5ex}{1.5ex}{1em}{1.6em}{.5em}{2.5em}{.2mm}

\def\nd*v{\rule[-\nd*depthdim]{\nd*thickness}{\nd*height}}
\def\nd*t{\rule[-\nd*depthdim]{0mm}{\nd*height}\rule[-\nd*depthdim]{\nd*thickness}{\nd*initheight}}
\def\nd*i{\hspace{\nd*indent}} 
\def\nd*s{\hspace{\nd*hsep}}
\def\nd*g#1{\nd*f{\makebox[\nd*indent][c]{$#1$}}}
\def\nd*f#1{\raisebox{0pt}[0pt][0pt]{$#1$}}
\def\nd*u#1{\makebox[0pt][l]{\settowidth{\nd*dim}{\nd*f{#1}}%
    \addtolength{\nd*dim}{2\nd*hsep}\hspace{-\nd*hsep}\rule[-\nd*depthdim]{\nd*dim}{\nd*thickness}}\nd*f{#1}}

% Lists

\def\nd*push#1#2{\expandafter\gdef\expandafter#1\expandafter%
  {\expandafter\nd*cons\expandafter{#1}{#2}}}
\def\nd*pop#1{{\def\nd*nil{\gdef#1{\nd*nil}}\def\nd*cons##1##2%
    {\gdef#1{##1}}#1}}
\def\nd*iter#1#2{{\def\nd*nil{}\def\nd*cons##1##2{##1#2{##2}}#1}}
\def\nd*modify#1#2#3{{\def\nd*nil{\gdef#1{\nd*nil}}\def\nd*cons##1##2%
    {\advance#2-1 ##1\advance#2 1 \ifnum#2=1\nd*push#1{#3}\else%
      \nd*push#1{##2}\fi}#1}}

\def\nd*cont#1{{\def\nd*t{\nd*v}\def\nd*v{\nd*v}\def\nd*g##1{\nd*i}%
    \def\nd*i{\nd*i}\def\nd*nil{\gdef#1{\nd*nil}}\def\nd*cons##1##2%
    {##1\expandafter\nd*push\expandafter#1\expandafter{##2}}#1}}

% Layer B

\newcount\nd*n
\def\nd*beginb{\begingroup\nd*reset\gdef\nd*stack{\nd*nil}\nd*push\nd*stack{\nd*t}%
  \begin{array}{l@{\hspace{\nd*labelsep}}l@{\hspace{\nd*justsep}}l}}
\def\nd*resumeb{\begingroup\begin{array}{l@{\hspace{\nd*labelsep}}l@{\hspace{\nd*justsep}}l}}
\def\nd*endb{\end{array}\endgroup}
\def\nd*hypob#1#2{\nd*f{\nd*num{#1}}&\nd*iter\nd*stack\relax\nd*cont\nd*stack\nd*s\nd*u{#2}&}
\def\nd*haveb#1#2{\nd*f{\nd*num{#1}}&\nd*iter\nd*stack\relax\nd*cont\nd*stack\nd*s\nd*f{#2}&}
\def\nd*havecontb#1#2{&\nd*iter\nd*stack\relax\nd*cont\nd*stack\nd*s\nd*f{\hspace{\ndindent}#2}&}
\def\nd*hypocontb#1#2{&\nd*iter\nd*stack\relax\nd*cont\nd*stack\nd*s\nd*u{\hspace{\ndindent}#2}&}

\def\nd*openb{\nd*push\nd*stack{\nd*i}\nd*push\nd*stack{\nd*t}}
\def\nd*closeb{\nd*pop\nd*stack\nd*pop\nd*stack}
\def\nd*guardb#1#2{\nd*n=#1\multiply\nd*n by 2 \nd*modify\nd*stack\nd*n{\nd*g{#2}}}

% Layer C

\def\nd*clr{\gdef\nd*cmd{}\gdef\nd*typ{\relax}}
\def\nd*sto#1#2#3{\gdef\nd*typ{#1}\gdef\nd*byt{}%
  \gdef\nd*cmd{\nd*typ{#2}{#3}\nd*byt\\}}
\def\nd*chtyp{\expandafter\ifx\nd*typ\nd*hypocontb\def\nd*typ{\nd*havecontb}\else\def\nd*typ{\nd*haveb}\fi}
\def\nd*hypoc#1#2{\nd*chtyp\nd*cmd\nd*sto{\nd*hypob}{#1}{#2}}
\def\nd*havec#1#2{\nd*cmd\nd*sto{\nd*haveb}{#1}{#2}}
\def\nd*hypocontc#1{\nd*chtyp\nd*cmd\nd*sto{\nd*hypocontb}{}{#1}}
\def\nd*havecontc#1{\nd*cmd\nd*sto{\nd*havecontb}{}{#1}}
\def\nd*by#1#2{\ifx\nd*x#2\nd*x\gdef\nd*byt{\mbox{#1}}\else\gdef\nd*byt{\mbox{#1, \ndref{#2}}}\fi}

% multi-line macros
\def\nd*mhypoc#1#2{\nd*mhypocA{#1}#2\\\nd*stop\\}
\def\nd*mhypocA#1#2\\{\nd*hypoc{#1}{#2}\nd*mhypocB}
\def\nd*mhypocB#1\\{\ifx\nd*stop#1\else\nd*hypocontc{#1}\expandafter\nd*mhypocB\fi}
\def\nd*mhavec#1#2{\nd*mhavecA{#1}#2\\\nd*stop\\}
\def\nd*mhavecA#1#2\\{\nd*havec{#1}{#2}\nd*mhavecB}
\def\nd*mhavecB#1\\{\ifx\nd*stop#1\else\nd*havecontc{#1}\expandafter\nd*mhavecB\fi}
\def\nd*mhypocontc#1{\nd*mhypocB#1\\\nd*stop\\}
\def\nd*mhavecontc#1{\nd*mhavecB#1\\\nd*stop\\}

\def\nd*beginc{\nd*beginb\nd*clr}
\def\nd*resumec{\nd*resumeb\nd*clr}
\def\nd*endc{\nd*cmd\nd*endb}
\def\nd*openc{\nd*cmd\nd*clr\nd*openb}
\def\nd*closec{\nd*cmd\nd*clr\nd*closeb}
\let\nd*guardc\nd*guardb

% Layer D

% macros with optional arguments spelled-out
\def\nd*hypod[#1][#2]#3[#4]#5{\ifx\relax#4\relax\else\nd*guardb{1}{#4}\fi\nd*mhypoc{#3}{#5}\nd*set{#1}{#2}}
\def\nd*haved[#1][#2]#3[#4]#5{\ifx\relax#4\relax\else\nd*guardb{1}{#4}\fi\nd*mhavec{#3}{#5}\nd*set{#1}{#2}}
\def\nd*havecont#1{\nd*mhavecontc{#1}}
\def\nd*hypocont#1{\nd*mhypocontc{#1}}
\def\nd*base{undefined}
\def\nd*opend[#1]#2{\nd*cmd\nd*clr\nd*openb\nd*guard{#1}#2}
\def\nd*close{\nd*cmd\nd*clr\nd*closeb}
\def\nd*guardd[#1]#2{\nd*guardb{#1}{#2}}

% Handling of optional arguments.

\def\nd*optarg#1#2#3{\ifx[#3\def\nd*c{#2#3}\else\def\nd*c{#2[#1]{#3}}\fi\nd*c}
\def\nd*optargg#1#2#3{\ifx[#3\def\nd*c{#1#3}\else\def\nd*c{#2{#3}}\fi\nd*c}

\def\nd*five#1{\nd*optargg{\nd*four{#1}}{\nd*two{#1}}}
\def\nd*four#1[#2]{\nd*optarg{0}{\nd*three{#1}[#2]}}
\def\nd*three#1[#2][#3]#4{\nd*optarg{}{#1[#2][#3]{#4}}}
\def\nd*two#1{\nd*three{#1}[\relax][]}

\def\nd*have{\nd*five{\nd*haved}}
\def\nd*hypo{\nd*five{\nd*hypod}}
\def\nd*open{\nd*optarg{}{\nd*opend}}
\def\nd*guard{\nd*optarg{1}{\nd*guardd}}

\def\nd*init{%
  \let\open\nd*open%
  \let\close\nd*close%
  \let\hypo\nd*hypo%
  \let\have\nd*have%
  \let\hypocont\nd*hypocont%
  \let\havecont\nd*havecont%
  \let\by\nd*by%
  \let\guard\nd*guard%
  \def\ii{\by{$\Rightarrow$I}}%
  \def\ie{\by{$\Rightarrow$E}}%
  \def\Ai{\by{$\forall$I}}%
  \def\Ae{\by{$\forall$E}}%
  \def\Ei{\by{$\exists$I}}%
  \def\Ee{\by{$\exists$E}}%
  \def\ai{\by{$\wedge$I}}%
  \def\ae{\by{$\wedge$E}}%
  \def\ai{\by{$\wedge$I}}%
  \def\ae{\by{$\wedge$E}}%
  \def\oi{\by{$\vee$I}}%
  \def\oe{\by{$\vee$E}}%
  \def\ni{\by{$\neg$I}}%
  \def\ne{\by{$\neg$E}}%
  \def\be{\by{$\bot$E}}%
  \def\nne{\by{$\neg\neg$E}}%
  \def\r{\by{R}}%
}

\newenvironment{nd}{\begingroup\nd*init\nd*beginc}{\nd*endc\endgroup}
\newenvironment{ndresume}{\begingroup\nd*init\nd*resumec}{\nd*endc\endgroup}

\catcode`\*=\nd*astcode

% End of file fitch.sty
\makeatother

%%%%%%%%%%%%%%%%%%%%%%%%%%%%%%%%%%%%%%%%%%%%%%%%%%%%%%%%%%%%%%%%%%%%%%
%%% JD'S MODIFICATIONS TO SELLINGER'S FITCH.STY FILE
%%% to conform to  Huth & Ryan's notation and nomenclature.
%%% All Sellinger licenses statements are maintained.
%%%%%%%%%%%%%%%%%%%%%%%%%%%%%%%%%%%%%%%%%%%%%%%%%%%%%%%%%%%%%%%%%%%%%%
% Macros for Fitch-style natural deduction. 
% Author: Jon Doyle, North Carolina State University
% Created: Aug 23, 2012
% Modified: Dec 5, 2015
% Copyright: (C) 2012-2015 Jon Doyle
% Filename: hrfitch.sty
% URL: http://www.csc.ncsu.edu/faculty/doyle/
%%%%%%%%%%%%%%%%%%%%%%%%%%%%%%%%%%%%%%%%%%%%%%%%%%%%%%%%%%%%%%%%%%%%%%
%%% Beginning of modifications to fitch.sty

%%% Begin the special treatment of *
{\chardef\x=\catcode`\*
\catcode`\*=11
\global\let\nd*astcode\x}
\catcode`\*=11

%%% Add handler for code lines
%\def\nd*code{\nd*five{\nd*haved}[][]{}[]\ndcodefont}

%%% Define the justification types
\def\nd*init{%
  \let\open\nd*open%
  \let\close\nd*close%
  \let\hypo\nd*hypo%
  \let\have\nd*have%
  \let\hypocont\nd*hypocont%
  \let\havecont\nd*havecont%
  \let\by\nd*by%
  \let\guard\nd*guard%
%%% NEW PROOF LINE TYPE
%%% Add commands for a line of code in program proofs
%%% Typesets code in bold typewriter font with no line number using
%%% the C level have-step continuation line command.
%%% Removes extra indentation inserted by \nd*havecontc
   \def\code##1{\nd*havecontc{\hspace{-\ndindent}\texttt{\textbf{##1}}}}
%%% To add later, if needed: continuation lines for code?
%%%%%%%%%%%%%%%%%%%%%%%%%%%%%%%%%%%%%%%%%%%%%%%%%%
%%% THIS DEFINITION IS UNCHANGED FROM fitch.sty
%%% Repetition rule
  \def\r{\by{R}}%
%%%%%%%%%%%%%%%%%%%%%%%%%%%%%%%%%%%%%%%%%%%%%%%%%%
%%% THESE JUSTIFICATIONS REDEFINE ONES IN fitch.sty.
%%% Sellinger uses uppercase I,E; Huth & Ryan use lowercase
%%% Sellinger uses $\Rightarrow$; Huth & Ryan use $\rightarrow$
%%% Conjuction rules
  \def\ai{\by{$\wedge$i}}%
  \def\ae{\by{$\wedge$e}}%
%%% Disjunction rules
  \def\oi{\by{$\vee$i}}%
  \def\oe{\by{$\vee$e}}%
%%% Implication rules
  \def\ii{\by{$\rightarrow$i}}%
  \def\ie{\by{$\rightarrow$e}}%
%%% Negation rules
  \def\ni{\by{$\neg$i}}%
  \def\ne{\by{$\neg$e}}%
%%% Double negation rule
  \def\nne{\by{$\neg\neg$e}}%
%%% Bottom rule
  \def\be{\by{$\bot$e}}%
%%% Universal rules
  \def\Ai{\by{$\forall$i}}%
  \def\Ae{\by{$\forall$e}}%
%%% Existential rules
  \def\Ei{\by{$\exists$i}}%
  \def\Ee{\by{$\exists$e}}%
%%%%%%%%%%%%%%%%%%%%%%%%%%%%%%%%%%%%%%%%%%%%%%%%%%
%%% THESE JUSTIFICATIONS ARE ADDITIONS TO fitch.sty
%%% Biimplication rules
  \def\bii{\by{$\leftrightarrow$i}}%
  \def\bie{\by{$\leftrightarrow$e}}%
%%% One-sided conjunction rules
  \def\aeone{\by{$\wedge$e$_1$}}%
  \def\aetwo{\by{$\wedge$e$_2$}}%
%%% One-sided disjunction rules
  \def\oione{\by{$\vee$i$_1$}}%
  \def\oitwo{\by{$\vee$i$_2$}}%
%%% Double negation derived rule
  \def\nni{\by{$\neg\neg$i}}%
%%% Copy rule (synonym of repetition rule R)
  \def\copy{\by{copy}}%     %JD does \copy conflict with something?
%%% Proof by contradiction
  \def\pbc{\by{PBC}}%
%%% Modus Tollens
  \def\mt{\by{MT}}%
%%% Law of the Excluded Middle
  \def\lem{\by{LEM}}%
%%% de Morgan's laws
  \def\DMna{\by{deMorgan}}%
  \def\DMno{\by{deMorgan}}%
%%% Distribution of conjunction over disjunction
  \def\distao{\by{Dist.\ $\mathord{\land}(\mathord{\lor})$}}%
%%% distribution of disjunction over conjunction
  \def\distoa{\by{Dist.\ $\mathord{\lor}(\mathord{\land})$}}%
%%% Negation of bottom rules
  \def\bni{\by{$\bot_\neg$i}}%
  \def\bne{\by{$\bot_\neg$e}}%
%%% Negation of top rules
  \def\tni{\by{$\top_\neg$i}}%
  \def\tne{\by{$\top_\neg$e}}%
%%% Nonlogical introduction rules
  \def\premise{\by{Premise}}%
  \def\assumption{\by{Assumption}}%
%%% Propositional ellipsis indicator
%%% This is not an inference rule, indicates omitted subproof
  \def\propom{\by{[Prop]}}%
  \def\Propositional{\by{Propositional}}%
%%%%%%%%%%%%%%%%%%%%%%%%%%%%%%%%%%%%%%%%%%%%%%%%%%
%%% HOARE PROGRAM LOGIC RULES
%%% Proof precondition (aka premise)
  \def\Precondition{\by{Precondition}}%
%%% Implication of base logic (arithmetic)
  \def\Implied{\by{Implied}}%
  \def\ImpliedAR{\by{$\turn_{\text{AR}}$}}%
%%% Assignment statement
  \def\Assignment{\by{Assignment}}%
%%% Command compositions
  \def\Composition{\by{Composition}}%
%%% Conditional statement
%%% Huth and Ryan use single labels for all parts of conditional rules
  \def\IfStatement{\by{If-Statement}}%
  \def\IfStatementone{\by{If-Statement-1}}%
  \def\IfStatementtwo{\by{If-Statement-2}}%
%%% New labels for different parts and different rules
%%% Precondition of full conditional (usually not necessary)
  \def\IfPre{\by{If-Precondition}}%
%%% Precondition of Then branch
  \def\ThenPre{\by{If-Then-Precondition}}%
%%% Precondition of Else branch
  \def\ElsePre{\by{If-Else-Precondition}}%
%%% Postcondition of full conditional
  \def\IfOne{\by{If-Statement-1}}%
  \def\IfTwo{\by{If-Statement-2}}%
%%% While loop rules
%%% Loop invariant and guard: precondition of the loop body
  \def\InvariantGuard{\by{Invariant Hyp. and Guard}}%
%%% Partial-correctness rule
  \def\PartialWhile{\by{Partial-While}}%
%%% Total-correctness rule
  \def\TotalWhile{\by{Total-While}}%
%%%%%%%%%%%%%%%%%%%%%%%%%%%%%%%%%%%%%%%%%%%%%%%%%%
%%% MODAL LOGIC RULES
%%% Box modality rules
  \def\Bi{\by{$\Box$i}}
  \def\Be{\by{$\Box$e}}
  \def\boxi{\by{$\Box$i}}%
  \def\boxe{\by{$\Box$e}}%
%%% K modality rules (Ki, Ke are synonyms for Box rules)
  \def\Ki{\by{$K$i}}%
  \def\Ke{\by{$K$e}}%
  \def\KT{\by{$KT$}}%
  \def\Kfour{\by{$K4$}}%
  \def\Kfive{\by{$K5$}}%
%%% Agent-indexed K modality rules
  % \def\Kii[#1]{\by{$K_{#1}$i}}%
  % \def\Kie[#1]{\by{$K_{#1}$e}}%
  % \def\KiT[#1]{\by{$KT_{#1}$}}%
  % \def\Kifour[#1]{\by{$K4_{#1}$}}%
  % \def\Kifive[#1]{\by{$K5_{#1}$}}%
%%% Everyone Knows rules
  \def\EKi{\by{$E$i}}%
  \def\EKe{\by{$E$e}}%
  \def\KE{\by{$KE$}}%
  \def\EK{\by{$EK$}}%
%%% Common Knowledge rules
  \def\CKi{\by{$C$i}}%
  \def\CKe{\by{$C$e}}%
  \def\CK{\by{$CK$}}%
  \def\CT{\by{$CT$}}%  % derived rule
  \def\Cfour{\by{$C4$}}%
  \def\Cfive{\by{$C5$}}%
}

%%% Endthe special treatment of *
\catcode`\*=\nd*astcode

%%%%%%%%%%%%%%%%%%%%%%%%%%%%%%%%%%%%%%%%%%%%%%%%%%%%%%%%%%%%%%%%%%%%%%
%%% END OF CSC 503 MACROS
%%%%%%%%%%%%%%%%%%%%%%%%%%%%%%%%%%%%%%%%%%%%%%%%%%%%%%%%%%%%%%%%%%%%%%
\endinput
